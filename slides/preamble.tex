\usepackage{%
	amsmath,
	appendixnumberbeamer,					% appendix doesn't appear in page count	
% 	beamerthemesplit,			% What for?
	booktabs,							% gives \toprule \midrule \bottomrule
	color,
	commath,							% \dif esp. PDE, ODE,
	csquotes,
	epstopdf,
	framed,
% 	movie15,
	longtable,
	makecell,
	marvosym,							% several symbols: eg. nice @
	multicol,
	multimedia,
	nicefrac,
	siunitx,							% \num{}
	soul,
	textpos,
	wrapfig,
	xmpmulti,
	xspace,
	}
%  ================================================
%  ================================================

% \newcommand*{\quelle}[1]{{\textcolor{gray}{\small#1}}}			% smaller font for refs
\newcommand*{\quelle}[1]{{\scriptsize#1}}			% smaller font for refs

\usepackage[export]{adjustbox}			% Positioning of the Twitter Symbol
%\usepackage[makeroom]{cancel}
\usepackage[framemethod=TikZ]{mdframed}


%  ================================================
%  =====================  TIKZ  ===================
%  ================================================

\usepackage{tikz}
\usepackage{pgfplots}
% \pgfplotsset{compat=1.11}										% compability layer
\pgfplotsset{compat=newest}										% compability layer

\usetikzlibrary{arrows}
\usetikzlibrary{positioning,calc}						% positioning: node mit einer Position angeben
% (xshift=...,yshift=...) ; calc : damit den Befehl \coordinate genutzt werden kann
\usetikzlibrary{fit,shapes}									% shapes:zur Erstellung von unterschiedlichen Formen
% (ellipse, diamond, circle) und fit: zum Einpassen von text oder Form in einer anderen Form
\usetikzlibrary{decorations.pathreplacing}

%  =============================================

\usepackage{sansmathaccent}
\pdfmapfile{+sansmathaccent.map}
\usetikzlibrary{shapes.geometric, arrows,chains}
\tikzset{
  startstop/.style={
    rectangle, 
    rounded corners,
    minimum width=3cm, 
    minimum height=1cm,
    align=center, 
    draw=black, 
    fill=red!30
    },
  startsleft/.style={
    rectangle, 
    rounded corners,
    minimum width=3cm, 
    minimum height=1cm,
    align=left, 
    draw=black, 
    fill=red!30
    },
  startsright/.style={
    rectangle, 
    rounded corners,
    minimum width=3cm, 
    minimum height=1cm,
    align=right, 
    draw=black, 
    fill=red!30
    },
  process/.style={
    rectangle, 
    minimum width=3cm, 
    minimum height=1cm, 
    align=center, 
    draw=black, 
    fill=blue!30
    },
  decision/.style={
    rectangle, 
    minimum width=3cm, 
    minimum height=1cm, align=center, 
    draw=black, 
    fill=green!30
    },
  arrow/.style={thick,->,>=stealth},
  dec/.style={
    ellipse, 
    align=center, 
    draw=black, 
    fill=green!30
    },
  font={\fontsize{9pt}{12}\selectfont}
}


%  ================================================
%  =============  LML BEAMER LAYOUT   =============
%  ================================================

\makeatletter
% All the following is NEW
\renewcommand{\@makefntext}[1]%
{\parindent 0em\everypar={\hangafter 1\hangindent  0em}\raggedright
\noindent\@makefnmark\hskip 1em\ignorespaces#1}
%%%%%%%%%%%%%%%%%%%%
\renewcommand<>\beamer@framefootnotetext[1]{%
  \global\setbox\beamer@footins\vbox{%
    \hsize0.5\framewidth%NEW
    \textwidth\hsize
    \columnwidth\hsize
    \unvbox\beamer@footins
    \reset@font\footnotesize
    \@parboxrestore
    \protected@edef\@currentlabel
         {\csname p@footnote\endcsname\@thefnmark}%
    \color@begingroup
      \uncover#2{\@makefntext{%
        \rule\z@\footnotesep\ignorespaces#1\@finalstrut\strutbox}}%
    \color@endgroup}}
\def\beamer@autobreakframebox{%
  \global\setbox\beamer@splitbox=\box\voidb@x%
  \ifbeamer@autobreak%
    % Ok, frame was overful -> split it!
    \setbox\@tempboxa=\vsplit\beamer@framebox to\beamer@autobreakfactor\textheight%
    \global\setbox\beamer@splitbox=\box\beamer@framebox%
    \@tempdima=\ht\beamer@splitbox%
    \ifdim\@tempdima<\beamer@autobreaklastheight%
      \global\beamer@autobreaklastheight=\@tempdima\relax%
    \else%
      \setbox\@tempboxa=\vbox{\unvbox\@tempboxa\unvbox\beamer@splitbox}%
      \global\setbox\beamer@splitbox=\box\voidb@x%
    \fi%
    \setbox\beamer@framebox=\vbox to\textheight{\unvbox\@tempboxa%
      \vskip\beamer@framebottomskipautobreak%
      \ifvoid\beamer@splitbox%
        \ifvoid\beamer@footins%
        \else%
          \begingroup
            \usebeamercolor*[fg]{footnote}%
            \footnoterule %
\setlength{\multicolsep}{0pt}%NEW
\begin{multicols}{2}%NEW
            \unvbox \beamer@footins%
\end{multicols}%NEW
            \global\setbox\beamer@footins=\box\voidb@x%
          \endgroup  
        \fi%
      \fi%
      \beamer@exitcode%
    }%
  \else%
    \setbox\beamer@framebox=\vbox to\textheight{\unvbox\beamer@framebox%
      \vskip\beamer@framebottomskip%
      \ifvoid\beamer@footins%
      \else%
        \begingroup
          \usebeamercolor*[fg]{footnote}%
          \footnoterule %
\setlength{\multicolsep}{0pt}%NEW
\begin{multicols}{2}%NEW
          \unvbox \beamer@footins %
\end{multicols}%NEW
          \global\setbox\beamer@footins=\box\voidb@x%
        \endgroup 
      \fi%
      \beamer@exitcode}%
    \global\setbox\beamer@footins=\box\voidb@x%
  \fi%
  }

\xdefinecolor{lightblue}{rgb}{0,200,255}
\setbeamercolor{important}{bg=lightblue,fg=red}

\newenvironment{myindentpar}[1]%
{\begin{list}{}%
    {\setlength{\leftmargin}{#1}}%
  \item[]%
}
{\end{list}}

\usetheme[hideothersubsections]{Hannover}

\newcommand\BackgroundPicture[1]{
\setbeamertemplate{background}{
\parbox[c][\paperheight]{\paperwidth}{
\vfill \hfill
\includegraphics[width=1\paperwidth,height=1\paperheight]{#1}
\hfill \vfill
}}}

% conundrum environment
\definecolor{col2}{rgb}{0.8,.2,.2} % lml blue
\newenvironment{conn}[1]
  {\begin{samepage}
  \begin{mdframed}[linecolor=col2,linewidth=2pt,roundcorner=10pt,frametitle={\underline{{\sc #1}}}]}
  {\end{mdframed}
  \end{samepage}}

% solution environment
\definecolor{col3}{rgb}{0,.27,.4375} % lml blue
\newenvironment{soln}[1]
  {\begin{samepage}
  \begin{mdframed}[linecolor=col3,linewidth=2pt,roundcorner=10pt,frametitle={\underline{{\sc #1}}}]}
  {\end{mdframed}
  \end{samepage}}


% ====================   LML COLORS   ======================

\definecolor{lmlblue}{RGB}{0,70,112}
% \definecolor{lmlblue}{RGB}{0,51,102}
\definecolor{lmlgrey}{RGB}{142,142,142}
\definecolor{grey}{RGB}{210,210,210}
\definecolor{ken}{RGB}{9,11,179}
\definecolor{ole}{RGB}{0,0,255}
\definecolor{fermat}{RGB}{50,150,50}
\definecolor{bernoulli}{RGB}{200,50,50}


\setbeamercolor{alerted text}{fg=lmlblue}
\setbeamercolor{background canvas}{bg=white}
\setbeamercolor{block body alerted}{fg=lmlblue}
\setbeamercolor{block body}{fg=lmlblue}
\setbeamercolor{block body example}{fg=lmlblue}

\setbeamercolor{block title alerted}{fg=lmlblue}
\setbeamercolor{block title}{fg=lmlblue}
\setbeamercolor{block title example}{fg=lmlblue}

\setbeamercolor{fine separation line}{fg=lmlgrey}
\setbeamercolor{frametitle}{fg=lmlblue}
\setbeamercolor{item projected}{fg=white, bg=lmlblue}
\setbeamercolor{normal text}{fg=black}

\setbeamercolor{palette sidebar primary}{fg=lmlgrey}
\setbeamercolor{palette sidebar quarternary}{fg=lmlgrey}
\setbeamercolor{palette sidebar secondary}{fg=lmlgrey}
\setbeamercolor{palette sidebar tertiary}{fg=lmlgrey}
\setbeamercolor{section in sidebar}{fg=white}
\setbeamercolor{section in sidebar shaded}{fg=grey}

\setbeamercolor{separation line}{fg=lmlgrey}

\setbeamercolor{sidebar}{bg=lmlgrey}
\setbeamercolor{sidebar}{parent=palette primary}

\setbeamercolor{structure}{bg=white, fg=lmlblue}
%structure changes color of title in sidebar.
\setbeamercolor{subsection in sidebar}{fg=white}
\setbeamercolor{subsection in sidebar shaded}{fg=grey}

\setbeamercolor{title}{fg=lmlblue}
\setbeamercolor{titlelike}{fg=lmlgrey}
\setbeamerfont{subtitle}{series=\mdseries}

\logo{\includegraphics[width=1.cm]{img/LML_logo_only.jpg}}

\AtBeginSection[]
{
% 	\begin{frame}{Agenda}
% 		\tableofcontents
%  	\end{frame}
%
%  \begin{frame}{Agenda}
% 	 \tableofcontents[currentsection,hideothersubsections]
%  \end{frame}
}

\setbeamertemplate{navigation symbols}{
% 	\insertshortauthor \hfill \insertshorttitle \hfill \insertshortdate \hfill%
%  	\insertframenavigationsymbol   										% from one slide to the next slide
%  	\insertsectionnavigationsymbol
% %  	\insertsubsectionnavigationsymbol
% %   \insertdocnavigationsymbol
% %   \insertbackfindforwardnavigationsymbol
 	\hspace{.5em}
	\usebeamercolor[black]{footline}
	\usebeamerfont{footline}
% 	\scriptsize
%   \ifnum\insertframenumber>1
%   \insertframenumber\,/\,\inserttotalframenumber\,
    \ifnum\insertframenumber>0
			\insertframenumber\,/\,\inserttotalframenumber%
		\else \hspace{1em}
    \fi%
}

% ====================================================================
% ====================================================================


\newcommand{\lmlblue}[1]{\textcolor{lmlblue}{#1}}		% shorthand for colored statements
\newcommand{\blue}[1]{\textcolor{blue}{#1}}		% shorthand for colored statements
\newcommand{\green}[1]{\textcolor{ForestGreen}{#1}}		% shorthand for colored statements
\newcommand{\red}[1]{\textcolor{red}{#1}}			% shorthand for colored statements
\newcommand{\gray}[1]{\textcolor{gray}{#1}}			% shorthand for colored statements


\newcommand{\EV}[1]{\ensuremath{\operatorname{E}\!\left[ #1 \right]}\xspace}	% Expectation Value Operator

\newcommand{\EA}[1]{\ensuremath{\left\langle #1 \right\rangle}\xspace}		% Ensemble Average
\newcommand{\EAgr}{\ensuremath{g_{\langle\rangle}}\xspace}	% Ensemble Average Growth Rate
\newcommand{\TA}[1]{\ensuremath{\overline{#1}}\xspace}									% Time Average
\newcommand{\TAgr}{\ensuremath{\TA{g}}\xspace}						% Time Average Growth Rate

\newcommand{\EEblue}{\lmlblue{EE}\xspace}
\newcommand{\EEcon}{\lmlblue{Ergodicity Economics}\xspace}

\newcommand{\etal}{\textit{et al.}\xspace}
\newcommand{\apriori}{\textit{a priori}\xspace}
\newcommand{\ia}{\textit{i.a.}\xspace}
\newcommand{\ie}{\textit{i.e.}\xspace}
\newcommand{\etc}{\textit{etc.}\xspace}
\newcommand{\eg}{\textit{e.g.}\xspace}
\newcommand{\cf}{\textit{cf.}\xspace}
\newcommand{\vs}{\textit{vs.}\xspace}
\newcommand{\vv}{\textit{v.v.}\xspace }

\newcommand{\nth}{$^{\text{th}}$}
\newcommand{\gt}{g_{\text{time}}}
\newcommand{\gtc}{g_{\text{time}}^{\text{co}}}
\newcommand{\gtn}{g_{\text{time}}^{\text{nc}}}
\newcommand{\be}{\begin{equation}}
\newcommand{\ee}{\end{equation}}
\newcommand{\bea}{\begin{eqnarray*}}
\newcommand{\eea}{\end{eqnarray*}}
\newcommand{\bc}{\begin{center}}
\newcommand{\ec}{\end{center}}
\newcommand{\bi}{\begin{itemize}}
\newcommand{\ei}{\end{itemize}}
\newcommand{\toinf}{{\rightarrow\infty}}

\newcommand{\err}[1]{\varepsilon\left[#1\right]}
\newcommand{\phat}{\hat{p}}
\newcommand{\nhat}{\hat{n}}

\newcommand{\D}{{\Delta}}
\newcommand{\Dx}{{\Delta x}}
\newcommand{\Dy}{{\Delta y}}
\newcommand{\Du}{{\Delta u}}
\newcommand{\Dt}{{\Delta t}}
\newcommand{\Dv}{{\Delta v}}
\newcommand{\DW}{{\Delta W}}
\newcommand{\DU}{{\Delta U}}
\newcommand{\du}{{\delta u}}
\newcommand{\dt}{{\delta t}}
\newcommand{\mur}{\mu_{\text{riskfree}}}
\newcommand{\mue}{\mu_{\text{excess}}}
% \newcommand{\gens}{g_{\ave{\,}}}

\newcommand{\ft}[1]{\frametitle{#1}}
\newcommand{\bq}{\begin{quote}}
\newcommand{\eq}{\end{quote}}
% \newcommand{\ww}[1]{\bq{\small\rm#1\\}\eq}

\newcommand{\Var}[1]{\text{Var}\left[#1\right]}
\newcommand{\var}[1]{\text{var}\left[#1\right]}
\newcommand{\prob}[1]{\mathcal{P}\left(#1\right)}
\newcommand{\lra}{\longrightarrow}
\newcommand{\eps}{\varepsilon}
% \newcommand{\ga}{g_\text{ave}}
%\newcommand{\gt}{g_\text{typ}}
% \newcommand{\gbar}{\bar{g}}


\newcommand{\nmax}{{n_{\text{max}}}\xspace}
\newcommand{\Ito}{It\^{o}}
\newcommand{\SP}{S{\&}P500}
\newcommand{\lopt}{\ensuremath{l_{\text{opt}}}\xspace}
\newcommand{\gest}{g_{\text{est}}\xspace}


\newcommand{\elabel}[1]{\label{eq:#1}}
\newcommand{\eref}[1]{Eq.~(\ref{eq:#1})}
\newcommand{\Eref}[1]{Equation~(\ref{eq:#1})}

\newcommand{\flabel}[1]{\label{fig:#1}}
\newcommand{\fref}[1]{Fig.~\ref{fig:#1}}
\newcommand{\Fref}[1]{Figure~\ref{fig:#1}}
% \newcommand{\person}[1]{\textsc{#1}}			% looks not good in beamer presentations

%  ========================================================================= 

\font\domino=domino										%  Dice symbols
\def\die#1{{\domino#1}}

\newcommand*\checked{\textcolor{ForestGreen}{\Large \checkmark}}

% ==================================================
% ===============  Biblatex Befehle   ==============

\usepackage[
autocite=footnote,								% Struktur von \autocite{key} 
backend=bibtex8,									% Backend Auswahl
citestyle=authoryear-comp,				% Zitierstil,
bibstyle=authoryear,							% LitVz Stil
dashed=false,											% Autorname in LitVz immer angeben kein --
sorting=nyt,											% Sortierung im LitVz
% sortcites=false,									% Sortierung bei mehreren Keys in cite-Befehl
autolang=hyphen,									% wirksam, wenn langid gesetzt wird
% language=autobib,								% im LitVz Ausgabe entspr. der Ursprungssprache
backref=true,											% give page of citation in references
hyperref=true,										% transform citations into clickable links
arxiv=abs,												% Link to arXiv abstract Seite
url=true,doi=true,eprint=true,		% print url,doi,eprint
useprefix=true,										% Berücksichtigung der von, van, de, etc.
mincrossrefs=0,										% Min. Anzahl ab der ein crossref auch ins LitVz kommt
maxcitenames=2,										% Ab wann wird et alii gesetzt
mincitenames=1,										% Auf wie viele wird mit et alii reduziert
maxbibnames=5,									% Autorenanzahl auf X begrenzt im LitVz
% minbibnames=10									% Angabe wie vieler Autoren nach Schnitt
% terseinits=true,									% Donald Ervin Knuth ‘D. E.’ by default as DE is true
% firstinits=true,									% alle Initialen
% sorting=false,
% labelnumber=true,
]{biblatex}

\makeatletter
\def\blx@maxline{77}
\makeatother

\addbibresource{../../LML_bibliography/bibliography}

\renewbibmacro{in:}{}					% print no 'In:' before the <Journal Name>'	
\DeclareFieldFormat{url}{\textsf{URL:}\space\url{#1}}
\DeclareFieldFormat{doi}{DOI:\href{https://doi.org/#1}{#1}}

%  ================================================
%  ==================   HYPERREF   ================
%  ================================================
	
\usepackage{hyperref}

\hypersetup{%
% 	pdftitle = {EE},
	pdfauthor = {Mark Kirstein},
	pdfsubject= {Ergodicity Economics}, 
% 	pdfkeywords = {Ensemble Average, Time Average, Non-Ergodicity, DTT, OT},
	baseurl = {http://lml.org.uk/people/mark-kirstein/},
%   draft,												% no hyperlinking
%   pdftex,
%   bookmarksopenlevel={2},
  breaklinks=true,
  colorlinks=true,
  urlcolor=blue,
  linkcolor=blue,
  citecolor=blue	
}
% \pdfinfo{/CreationDate (D:20120210100000)}
% \pdfinfo{/ModDate (D:20120210100000)}

%  ================================================
%  =========   LOAD ONLY AFTER HYPERREF   =========
%  ================================================

\usepackage[anythingbreaks]{breakurl}